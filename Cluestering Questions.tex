\documentclass{article}
\usepackage{amsmath}
\usepackage{amsfonts}
\usepackage{amssymb}
\usepackage{enumerate}
\usepackage{graphicx}
\usepackage{geometry}
\geometry{a4paper, margin=1in}

\title{Numerical Questions on Hierarchical Clustering}
\author{}
\date{}

\begin{document}

\maketitle

\section*{Numerical Questions on Hierarchical Clustering}

\begin{enumerate}[1.]
    \item \textbf{Distance Calculation}:  
    Given the following points in a 2D space:
    \begin{itemize}
        \item Point A: (1, 2)
        \item Point B: (3, 4)
        \item Point C: (6, 8)
    \end{itemize}
    Calculate the Euclidean distance between:
    \begin{enumerate}[(a)]
        \item Point A and Point B  
        \item Point A and Point C  
        \item Point B and Point C  
    \end{enumerate}

    \item \textbf{Single-Linkage Clustering}:  
    Using the distances calculated in Question 1, apply single-linkage clustering:
    \begin{enumerate}[(a)]
        \item If you treat each point as a separate cluster, show how clusters would merge step-by-step until all points belong to one cluster.  
        \item At which step does the first merge occur, and which points are merged?
    \end{enumerate}

    \item \textbf{Complete-Linkage Clustering}:  
    Using the same set of points from Question 1, perform complete-linkage clustering:
    \begin{enumerate}[(a)]
        \item Determine the first two clusters that would merge based on the maximum distances calculated from the points.  
        \item What would the distance be between these clusters?
    \end{enumerate}

    \item \textbf{Group Average Linkage}:  
    Suppose we have two clusters formed after the first two merges in a hierarchical clustering process:
    \begin{itemize}
        \item Cluster 1: {A, B} with points A (1, 2) and B (3, 4)  
        \item Cluster 2: {C} with point C (6, 8)
    \end{itemize}
    Calculate the average distance between the points in Cluster 1 and Cluster 2.  
    Use the Euclidean distance as your measure.

    \item \textbf{Dendrogram Construction}:  
    Based on the following distances between points A, B, and C:
    \begin{itemize}
        \item Distance (A, B) = 2.24  
        \item Distance (A, C) = 7.21  
        \item Distance (B, C) = 4.47  
    \end{itemize}
    Construct a simple dendrogram that shows how these points would be clustered.  
    Indicate the height at which each merge occurs.

    \item \textbf{Silhouette Score Calculation}:  
    Consider three clusters with the following average distances:
    \begin{itemize}
        \item Cluster 1: Average distance within the cluster = 0.5  
        \item Cluster 2: Average distance within the cluster = 1.2  
        \item Cluster 3: Average distance to the nearest cluster (inter-cluster distance) = 1.5
    \end{itemize}
    Calculate the silhouette score for Cluster 1 using the formula:
    \[
    \text{Silhouette Score} = \frac{b - a}{\max(a, b)}
    \]
    where \( a \) is the average distance to points in the same cluster and \( b \) is the average distance to the nearest cluster.

    \item \textbf{Cophenetic Correlation Coefficient}:  
    Suppose you computed the cophenetic distances for a hierarchical clustering and obtained the following distances:
    \begin{itemize}
        \item Observed distances: [2, 3, 5]  
        \item Cophenetic distances: [2.2, 3.1, 5.5]  
    \end{itemize}
    Calculate the cophenetic correlation coefficient using the formula:
    \[
    r = \frac{\text{Cov}(\text{observed}, \text{cophenetic})}{\sigma_{\text{observed}} \sigma_{\text{cophenetic}}}
    \]
    (You can assume hypothetical values for the covariance and standard deviations for this calculation if needed).

    \item \textbf{Non-Monotonicity Example}:  
    Given a hypothetical dendrogram showing merges of three clusters (A, B, C) at different heights, analyze whether the dendrogram violates monotonicity.  
    \begin{itemize}
        \item List the merged clusters and their respective heights.  
        \item Identify if any clusters at higher levels are more similar than those at lower levels.
    \end{itemize}
\end{enumerate}

\newpage

\section*{Answers to Selected Questions}

1. \textbf{Distance Calculation}:
    \begin{enumerate}[(a)]
        \item \( d(A, B) = \sqrt{(3-1)^2 + (4-2)^2} = \sqrt{2^2 + 2^2} = \sqrt{8} \approx 2.83 \)  
        \item \( d(A, C) = \sqrt{(6-1)^2 + (8-2)^2} = \sqrt{5^2 + 6^2} = \sqrt{25 + 36} = \sqrt{61} \approx 7.81 \)  
        \item \( d(B, C) = \sqrt{(6-3)^2 + (8-4)^2} = \sqrt{3^2 + 4^2} = \sqrt{9 + 16} = \sqrt{25} = 5 \)
    \end{enumerate}

2. \textbf{Single-Linkage Clustering}:
    \begin{enumerate}[(a)]
        \item First merge: A and B (distance = 2.83).  
        Next merge: (A, B) with C (distance = 5).
        \item First merge occurs between A and B.
    \end{enumerate}

3. \textbf{Complete-Linkage Clustering}:
    \begin{enumerate}[(a)]
        \item First merge: A and B (distance = 2.83).  
        Next merge: (A, B) with C (distance = 7.81).
        \item The distance between the merged clusters is 7.81.
    \end{enumerate}

4. \textbf{Group Average Linkage}:
    \begin{itemize}
        \item Average distance between Cluster 1 (A, B) and Cluster 2 (C):
        \[
        \text{Average distance} = \frac{d(A,C) + d(B,C)}{2} = \frac{7.81 + 5}{2} \approx 6.41
        \]
    \end{itemize}

5. \textbf{Dendrogram Construction}:
    \begin{itemize}
        \item Merges would occur at:
        \begin{itemize}
            \item A and B at height 2.24.
            \item A, B and C at height 4.47.
        \end{itemize}
    \end{itemize}

6. \textbf{Silhouette Score Calculation}:
    \begin{itemize}
        \item For Cluster 1:  
        \( a = 0.5 \), \( b = 1.5 \)  
        \[
        \text{Silhouette Score} = \frac{1.5 - 0.5}{\max(0.5, 1.5)} = \frac{1}{1.5} \approx 0.67
        \]
    \end{itemize}

7. \textbf{Cophenetic Correlation Coefficient}:
    \begin{itemize}
        \item Calculate \( r \) with assumed covariance and standard deviations for observed and cophenetic distances.
    \end{itemize}

8. \textbf{Non-Monotonicity Example}:
    \begin{itemize}
        \item Analyze and provide reasoning based on the heights and similarities presented.
    \end{itemize}

\end{document}
